\chapter{Introduction}
	\textbf{Mechatronics} is the clever integration of three branches of engineering: mechanics, electronics and control. The use of mechatronics has modified the way to design objects, and during the design of a machine there's also to keep in mind many other factor, like the financial and manufacturing technologies.
	
	In general the result of the design should be simpler, cheaper and more reliable but also trying to increase performances; this simplification isn't associated to a easier mechanical approach, but moreover of a more complex one. To be successful a mechatronic approach needs to be established from the very earliest stages og the conceptual design process where option can be kept open before the form of the embodiment is determined.
	
	The skills needed to design this type of system are the mathematical modelling, the knowledge of different actuators and sensors, but also microprocessor, control algorithm and mechanical design. It's possible to use a modular approach, dividing the parts of the machine into simpler element (the mechanical or control one, for example). In particular the mechanical realisation combines the assembly module with the actuation and measurement modules, so the object that interact with the world.